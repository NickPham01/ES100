
The sound of an electric guitar is a complex combination of processes.  Unlike a purely acoustic instrument, timbre is not purely a result of the vibration of the wood and strings.  The pickups, amplifier, and speaker are all in a sense part of the instrument.  In addition to these standard components, many electric guitarists use one or more effect pedals to augment or alter their sound. First developed in the 1960s, there are now thousands of pedals available for guitarists to use.

\section{Current Guitar Effects Pedal Auditioning Methods}

The wide variety of options for guitarists to choose from means that it can be difficult to decide which product to purchase.  Though in an age of Internet shopping there are many guitar related products available for purchase on-line, most players want to test potential purchases in person before buying them.  Subtle differences in sound and feel make this live testing paramount.  As such, brick-and-mortar music stores such as Guitar Center still have a place in the industry.  


\subsection{Permanent Display Board}

There are two existing methods for testing effects pedals at brick-and-mortar retail stores such as Guitar Center.  The first one that many guitarists experience is the permanent display board.  While most of the in-stock effects pedals at retailers like Guitar Center are stored in a general display case for customers to peruse, some larger companies have dedicated, integrated display boards to showcase their products.  For example, Figure \ref{fig:BossDisplay} shows one such display, containing products from Boss, one of the most popular companies in the industry.  As can be seen, the pedals are permanently attached to the display board, and are connected in series, with power available to all effects at once.  The last pedal in series remains connected to a dedicated guitar amp, so customers must only plug a guitar into the input of the display unit to begin auditioning.  This facilitates easy switching between any of the pedals that are included in the display.  Much like they would when using the effects in their own pedalboard after purchase, the user can select which effects are engaged simply by activating the foot switch on each pedal.  This makes it quick and easy to switch between different effects.

However, this system has some limitations.  The first is that it is generally limits a customer to testing pedals from a single manufacturer at any time.  This is especially important if they are attempting to decide between several similar options from competing companies.  The second issue is that the order of connection of the effects is fixed.  In general, changing the connection topology of effects can result in radical differences in the output, which can affect the perceived quality of a pedals sound.  For example, consider the different possible methods of connecting a distortion and delay pedal.  As the distortion effect is typically the result of signal gain and clipping, it is inherently non-linear, which helps explain the differences in sound for different connections.  If the distortion is connected in series before the delay pedal, the sound might be the main distorted guitar sound with some echoes of this distorted sound.  If the guitar is first connected to the delay instead, the echoes will be all the same volume as the gain and clipping of the distortion will even out the amplitude.  In theory, it is also possible to connect the two pedals in parallel, which would result in the main distorted guitar sound and echoes of the "clean", unaffected guitar.  A feedback connection where the distortion pedal is connected in a feedback loop of the delay might also produce interesting results.  As can be seen even with this simple example involving only two effects, there are a multitude of possible connection topologies available to guitarists if they are not limited by the fixed connections of display boards.

The last major issue stems from the bypass method used by many of these units.  Many guitar pedals, include all of the products offered by Boss, include a buffer that is active when the main "effect" is bypassed.  This is useful for performing guitarists who often use cables totaling up to fifty feet in length.  The capacitive loading from the coaxial instrument cables can result in high frequency loss, as guitar pickups can have DC output resistance on the order of $10k\Omega$.  A high input impedance and low output impedance buffer located in a pedal can help reduce the effects of this typically undesirable high frequency attenuation.  However, because these display boards can contain thirty or more effects, there can be issues with signal degradation when so many buffers are connected in series.

In some cases, these buffers can have an additional consequence that can be undesirable for certain types of pedals with low input impedances.  Effects like the "Fuzz Face", made famous for its use by Jimi Hendrix, can have an input impedance on the order of $10k\Omega$, which is similar to the guitar's output impedance.  This means that there can be interactions between the two circuit blocks, which can be subject to change based on the position of the guitar's on-board controls as well.  Though in typical engineering applications interactions between circuit blocks is undesirable, many guitarists prize this "interactiveness" as a benefit.  Placing a buffer between the output of the guitar and the input of the pedal disrupts these interactions.

\insertimage{0.5}{PR2Images/BossDisplay.jpg}{A permanent display board typical to those found at retail stores such as Guitar Center.  This particular example is from Boss, one of largest guitar pedal manufacturers, and one of the most ubiquitous companies to use these display boards.}{fig:BossDisplay}

\subsection{Free Connection Method}

The other common method guitarists use to test effects pedals will be referred to as a "free connection" method.  If a customer wishes to compare two pedals, they must request an employee to bring the particular units to a demo table, as seen in Figure \ref{fig:GCtestbench}.  To perform an A-B test, the customer must turn the amplifier output off to prevent pops before disconnected the current pedal being tested and replacing it with the other, if they wish to avoid issues with buffers mentioned above.  This time required for swapping the pedals limits the customer's ability to accurately judge the relative qualities of the units under test.  This cost can result in several negative consequences.  First, because of the reduction in accuracy of the A-B test, the guitarist may not be able to make a well-informed decision as to which effect best suits them, which could result in them either purchasing an inferior product, or in them deciding not to purchase any product at all.  Second, the known difficulties of testing effects pedals in this way may deter potential customers from even bothering at all.  This would either limit people from making "impulsive" purchases or encourage people to buy both of the products they are interested in, test them at home, and return the one they did not prefer.  The latter is inefficient for retail stores to deal with many returns and refunds.  Despite these drawbacks, this method is preferred if pedals from multiple manufacturers are being compared or if the routing configurations being tested are more complex (i.e. the user wants to test two pedals in different orders).

\insertimage{0.5}{PR2Images/GuitarCenterBoard.jpg}{A typical "test bench" demo board found at Guitar Center, Boston MA.  The surface is carpeted to avoid scratching the pedals.  Sitting on the surface is a portable type of display board.  Several cables and power supplies can be seen.  The amplifier used for testing is located below the table on which the demo board rests.}{fig:GCtestbench}