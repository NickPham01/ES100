Taking a look back at the results of the measurements performed indicates that this project, has fulfilled at least some aspects of each design goal set out in the problem definition.  In the functional improvements category, the 65\% improvement in swap time over the free connection method shows promise.  There is room for future work in redesigning the mechanical features of the hot swapping receiver to guide the plate into the correct alignment, which will further reduce the swap time for this method.  With this modification, the solution will likely meet its factor of five improvement goal.

The compatibility requirement was exceeded, with about 85\% market coverage.  However, here too there is opportunity for further work.  An special plate designed for over-sized pedals that uses two hot swapping receivers would allow for even the largest of effects like TC Electronics' 4-series products to be integrated with this system.

The signal quality tests performed indicated that the device on its own had excellent noise and frequency characteristics.  However, issues with the power supply hampered the effectiveness of this both during testing and the live demonstration.  A new power supply should be chosen to mitigate this issue.

Although the relay-based design of this system ensured there were no additional transients added to the signals during switching, an additional direction for future work would be a muting circuit that damped the output during switching to reduce large transients as a result of instantaneous differences between the signals being switched.  Care must be taken to ensure that this muting system does not load the circuit, reduce the signal quality that so much work has gone into to maintain, or load down the signal.

A final modification in the future would be to provide protection for those pedals which can accept no more than 9VDC at their inputs.  Because the current method allows customers to select the desired voltage, they might accidentally select 18V power for a 9V max pedal which in the worse case could cause catastrophic failure of the pedal.  A method to prevent this must still allow the user to select different voltages when desired.  

Though designed for helping customers test guitar pedals in the store, it has several other potential applications, including for studio musicians who might use it to quickly create new, interesting sounds from their own collection of pedals.  At the end of the day, this solution combined the best attributes of both the free connection and display board methods of auditioning guitar pedals.  In addition, it added the great ability to test effects in various routings, which was much more difficult in previous methods.